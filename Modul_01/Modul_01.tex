\documentclass[a4paper,11pt]{extarticle}
\usepackage[a4paper]{geometry}
\geometry{verbose,tmargin=2cm,bmargin=2cm,lmargin=2cm,rmargin=2cm}

\usepackage{fontspec}
%\setmainfont{DejaVu Serif}
%\setsansfont{DejaVu Sans}
%\setmonofont{DejaVu Sans Mono}
%\setmonofont{Menlo}
%\setmainfont{FreeSerif}
%\setsansfont{FreeSans}
\setmonofont{FreeMono}

\setlength{\parindent}{0cm}
\setlength{\parskip}{0.5em}

\usepackage{textcomp}

\usepackage{hyperref}
\usepackage{url}
\usepackage{xcolor}

\usepackage{minted}
\newminted{cpp}{breaklines,fontsize=\small}
\newminted{text}{breaklines,fontsize=\small}

\definecolor{mintedbg}{rgb}{0.95,0.95,0.95}
\usepackage{mdframed}

\BeforeBeginEnvironment{minted}{\begin{mdframed}[backgroundcolor=mintedbg]}
\AfterEndEnvironment{minted}{\end{mdframed}}

\title{
MI2102\\
Praktikum Teknik Komputasi\\
Modul 1}
\author{Fadjar Fathurrahman}
\date{2018}

\begin{document}
\maketitle

\section{Tujuan}
\begin{itemize}
\item Dapat membuat project C/C++ sederhana dengan menggunakan CodeBlocks
\item Dapat membuat kode sumber C/C++ dengan menggunakan CodeBlocks
  atau editor teks
\item Dapat melakukan proses build project C/C++ pada CodeBlocks
\item Mengenal baris perintah (*command line* pada Linux) serta beberapa perintah sederhana
\item Dapat melakukan kompilasi program dengan menggunakan baris perintah pada Linux
\item Mengenal beberapa kesalahan umum pada pembuatan program serta solusinya
\item Dapat membuat program C++ sederhana untuk melakukan perhitungan matematika
sederhana
\end{itemize}

\section{Perangkat lunak yang diperlukan}
\begin{itemize}
\item Linux OS
\item CodeBlocks yang telah dikonfigurasi untuk kompiler GNU C/C++
\item Terminal emulator dengan *bash* sebagai shell (baris perintah)
\item Editor teks seperti \texttt{gedit}
\end{itemize}


\section{Petunjuk dasar}

Buat direktori (folder) dengan format \texttt{Modul01\_NIM\_Nama}. Semua
pekerjaan Anda akan dilakukan dalam direktori ini.

\textbf{Jangan gunakan spasi jika nama Anda lebih dari satu kata}

\textbf{Gunakan underscore \_ sebagai pengganti spasi}

Contoh:
\begin{minted}[fontsize=\small]{text}
Modul01_Jaka_Sembung (benar)

Modul01_Jaka Sembung (salah)
Modul01 Jaka Sembung (salah)
\end{minted}

\section{Program "Hello World" dengan menggunakan CodeBlocks}

Buatlan sebuah proyek kosong pada CodeBlocks dengan nama \texttt{Hello01}
dengan langkah sebagai berikut:
\begin{enumerate}
\item Pada menu \textsf{New} \textrightarrow \textsf{Project}
      \textrightarrow \textsf{Empty Project}
%
\item Konfigurasi project:
  \begin{itemize}
  \item \textsf{Project Title}: Hello01
  \item \textsf{Folder to create project in}: \texttt{/home/students/Modul01\_Jaka\_Sembung/}
  \item \textsf{Project filename}: Hello01.cbp
  \item \textsf{Resulting filename}: \\
  \texttt{/home/students/Modul01\_Jaka\_Sembung/Hello01/Hello01.cbp}
  \item Kemudian klik \textsf{Next}
  \end{itemize}
  \textsf{Next}.
%
\item Pastikan bahwa yang digunakan adalah \textsf{Compiler GCC}
ceklis semua, jangan ubah konfigurasi. Kemudian \textsf{Finish}
%
\item Tambahkan file baru: menu 
      \textsf{New} \textrightarrow \textsf{Files} \textrightarrow
      \textsf{Category C/C++ source}
      \textrightarrow \textsf{C++} \textrightarrow
      Masukkan path file \textrightarrow
      \textrightarrow{Add file to active projects}.
      Ceklis semua pilihan \textsf{Debug/Release} dan \textsf{Finish}.
      Nama file: \textsf{Hello01.cpp} dengan isi sebagai berikut
%
\begin{minted}[fontsize=\small]{cpp}
// Nama:
// NIM:

#include <iostream>
using namespace std;

int main()
{
  cout << "Hello World" << endl;
  return 0;
}
\end{minted}
%
\end{enumerate}

Kemudian \textsf{Build (F9)}. Pastikan bahwa tidak ada kesalahan pada saat proses ini.

\subsection{Tugas}

\begin{enumerate}
\item Tuliskan output dari program ketika dijalankan.
%
\item Tuliskan output dari \textsf{Build Log} dan \textsf{Build messages}.
%
\item Berikut ini adalah struktur dari direktori \texttt{Hello01} ketika Anda
      telah berhasil membangun dan menjalankan proyek \textsf{Hello01}
      \footnote{Pada beberapa kasus, file \texttt{*.depend} dan \texttt{*.layout} tidak
      ada pada direktori Anda. Tidak perlu khawatir karena hal tersebut tidak penting
      dalam praktikum ini.}:
%
\begin{minted}[fontsize=\small]{text}
Hello01
├── bin
│   └── Debug
│       └── Hello01
├── Hello01.cbp
├── Hello01.cpp
├── Hello01.depend
├── Hello01.layout
└── obj
    └── Debug
        └── Hello01.o
\end{minted}
   Buat file baru dengan nama \texttt{Hello02.cpp} dalam direktori ini sehingga
   struktur direktorinya menjadi sebagai berikut:
%
\begin{minted}[fontsize=\small]{text}
Hello01/
├── bin
│   └── Debug
│       └── Hello01
├── Hello01.cbp
├── Hello01.cpp
├── Hello01.depend
├── Hello01.layout
├── Hello02.cpp    <--- File yang baru dibuat
└── obj
    └── Debug
        └── Hello01.o
\end{minted}
%
  Isi file \textsf{Hello02.cpp} ini sebagai berikut:
%
\begin{minted}[fontsize=\small]{cpp}
#include <iostream>
using namespace std;
int main()
{
  cout << "Hello this is MI2101" << endl;
  return 0;
}
\end{minted}
%
   Kemudian coba build kembali project yang Anda buat. Apa yang terjadi ?
   Tuliskan keluaran dari program apabila project ini berhasil dibangun.
   Jika terjadi kesalahan, apa yang harus dilakukan?
%
\item Tutup semua jendela CodeBlocks.
      Sekarang buka File Explorer di Ubuntu, kemudian buka direktori di mana
      project CodeBlocks ini Anda buat.
      %
      Bandingkan dua kasus berikut.
      \begin{itemize}  
      \item CodeBlocks digunakan untuk membuka file \texttt{Hello01.cpp}
      \item CodeBlocks digunakan untuk membuka file \texttt{Hello01.cbp}
      \end{itemize}
      (Untuk masing-masing kasus CodeBlocks harus ditutup terlebih dahulu)
      %
      Adakah perbedaan yang Anda amati? Apakah perbedaan antara \texttt{Hello01.cpp}
      dengan \texttt{Hello01.cbp} ?

\end{enumerate}

\section{Program 'Hello World' dengan menggunakan baris perintah}

Buka Terminal (program khusus untuk menjalankan baris perintah).

Perhatikan demonstrasi yang dilakukan di depan kelas.

Catat perintah-perintah yang telah didemonstrasikan dalam tabel berikut ini.

{\centering
\begin{tabular}{|p{0.4\textwidth}|p{0.6\textwidth}|}
\hline
Perintah & Penjelasan mengenai perintah (keluaran dan/atau apa yang dilakukan) \\
\hline
\texttt{ls} & \\
\texttt{pwd} & \\
dan seterusnya & \\
\hline
\end{tabular}
}

Lengkapi langkah-langkah (dan perintah yang digunakan pada terminal) berikut ini yang
diperlukan untuk menjalankan program \texttt{Hello01}. Kita akan menggunakan
file \texttt{Hello01.cpp} yang sudah kita buat pada CodeBlocks sebelumnya.

\begin{enumerate}
\item Cek direktori kerja awal ketika membuka terminal.
   Perintah yang diperlukan adalah ....
   Direktori kerja awal adalah ....
%
\item Tentukan lokasi file \texttt{.cpp} yang ingin dibangun
   (\textsf{build = compile + link}). Lokasi file tersebut adalah ....
%
\item Ubah direktori kerja ke direktori yang berisi file \texttt{.cpp} yang telah
   disebutkan sebelumnya.
   Direktori tersebut adalah ....
   Perintah yang diperlukan adalah ....
%
\item Pastikan bahwa perintah yang dilakukan pada langkah 3 sukses dan kita
   tidak melakukan kesalah dengan cara mengecek direktori kerja sekarang.
   Perintah yang diperukan adalah ....
   Direktori kerja sekarang adalah ....
%
\item Pastikan bahwa file \texttt{.cpp} tersebut ada pada direktori kerja.
   Hal ini dapat dilakukan dengan cara melihat daftar file dan subdirektori apa
   saja yang ada pada direktori saat ini.
   Perintah yang diperlukan adalah ....
   File-file yang ada pada direktori kerja adalah ....
%
\item Bangun (\textsf{Build}) program dengan menggunakan perintah ....
%
\item Pastikan proses pada langkan sebelumnya telah berhasil (tidak ada kesalahan atau pesan error).
   Cek apakah ada file baru yang dihasilkan.
   Jika ada file tersebut bernama ....
%
\item Eksekusi program yang dihasilkan pada terminal.
   %
   Perintah yang diperlukan adalah ....
    %
   Hasil/output dari eksekusi program
   %
   ....
\end{enumerate}


\section{Perhitungan arimatika sederhana}

Buat project CodeBlocks baru dengan nama \textsf{Aritmatika01} dan tambahkan file
C++ pada proyek tersebut dengan nama \textsf{aritmatika01.cpp}.

\textbf{Alternatif} Untuk Anda yang lebih senang bekerja dengan
terminal, buat direktori \textsf{Aritmatika01} dan file \texttt{aritmatika01.cpp}.

Berikut ini adalah kode sumber untuk \texttt{aritmatika01.cpp}.

\begin{minted}[fontsize=\small]{cpp}
// Nama:
// NIM:
#include <iostream>
using namespace std;

int main()
{
  int a, b, c;

  a = 123;
  b = 456;

  // Tampilkan nilai a and b ke layar
  cout << "a = " << a << endl;
  cout << "b = " << b << endl;

  // Lakukan perhitungan tambah, kurang, kali, dan bagi
  c = a + b;
  cout << "a + b = " << c << endl;
  
  c = a - b;
  cout << "a - b = " << c << endl;

  c = a * b;
  cout << "a * b = " << c << endl;

  c = a / b;
  cout << "a / b = " << c << endl;

  return 0;
}
\end{minted}

\subsection{Tugas}
\begin{enumerate}
\item Tuliskan keluaran dari program tersebut ketika dijalankan.
%
\item Apakah ada hasil perhitungan yang Anda anggap salah ?
      Bagaimana cara memperbaikinya ?
\end{enumerate}


\section{Latihan dan Eksplorasi (Bagian A)}

Pada kerjakan latihan ini dengan menggunakan CodeBlocks atau baris perintah
pada terminal.

Untuk setiap percobaan, 
serta berikan berikan penjelasan dan/atau komentar Anda.

Cek apakah kode berikut ini dapat dibuild atau dijalankan tanpa
ada kesalahan (\textit{syntax error}).

Tuliskan pesan kesalahan (jika ada) yang tampil pada \textsf{Build Log}
(jika menggunakan CodeBlocks) atau terminal, jelaskan kesalahan
yang terjadi kemudian perbaiki.

1. Kode sumber
\begin{cppcode}
#include <iostream>
using namespace std;
int main()
{
  cout << "Hello World" << endl
  return 0;
}
\end{cppcode}

2. Kode sumber
\begin{cppcode}
#include <iostream>
using namespace std;
int main()
{
  cout < "Hello World" < endl
  return 0;
}
\end{cppcode}

3. Kode sumber
\begin{cppcode}
using namespace std;
int main()
{
  cout << "Hello World" << endl;
  return 0;
}
\end{cppcode}


4. Kode sumber:
\begin{cppcode}
#include <iostream>
int main()
{
  cout << "Hello World" << endl;
  return 0;
}
\end{cppcode}

5. Kode sumber:
\begin{cppcode}
#include <iostream>
int main()
{
  std::cout << "Hello World" << std::endl;
  return 0;
}
\end{cppcode}


6. Kode sumber:
\begin{cppcode}
#include <iostream>
using namespace std;
int main()
{
  cout << "Hello World" << endl;
}
\end{cppcode}

Referensi berikut ini boleh dibaca:
\begin{itemize}
\item \url{https://www.codeproject.com/Questions/693038/why-do-we-have-to-use-return}
\item \url{http://www.cplusplus.com/forum/beginner/24461/}
\end{itemize}

7. Kode sumber
\begin{cppcode}
#include <stdio.h>
int main()
{
  printf("Hello World\n");
  return 0;
}
\end{cppcode}

8. Kode sumber
\begin{cppcode}
#include <cstdio>
int main()
{
  printf("Hello World\n");
  return 0;
}
\end{cppcode}


\section{Latihan dan eksplorasi (Bagian B)}

\begin{enumerate}
\item Modifikasi program pada proyek \textsf{Aritmatika01} agar menghasilkan
   output sebagai berikut (dengan menggunakan nilai a dan b yang sama)
%
\begin{textcode}
123 + 456 = 579
123 - 456 = -333
123 * 456 = 56088
123 / 456 = 0.269737
\end{textcode}
%
\item Lengkapi program berikut ini. Program ini bertujuan untuk mencari akar
   dari persamaan kuadrat $ax^2 + bx + c = 0$, yaitu:
   \begin{equation}
   x_{1,2} = \frac{-b \pm \sqrt{b^2 - 4ac}}{2a}
   \end{equation}
   %
  Untuk sementara, program ini hanya memperhitungkan kasus nilai diskriminan
  $D = b^2 - 4ac \geq 0$.
  Kasus yang lebih umum akan dipelajari di Modul 2.
%
\begin{cppcode}
#include <iostream>
#include <cmath>  // diperlukan untuk sqrt
using namespace std;
int main()
{
  ....
  
  a =  1.2;
  b =  1.3;
  c = -4.0
  
  // Hitung diskriminan
  .... d;
  d = ....;
  // Tampilkan nilai diskriminan
  cout << ....

  float x1, x2;
  x1 = ....;
  x2 = ....;

  // Tampilkan nilai x1 dan x2
  cout << "x1 = " << x1 << endl;
  cout << "x2 = " << x2 << endl;

  return 0;
}
\end{cppcode}
%
\end{enumerate}


\end{document}
