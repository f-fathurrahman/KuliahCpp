\documentclass[a4paper,11pt]{extarticle}
\usepackage[a4paper]{geometry}
\geometry{verbose,tmargin=2cm,bmargin=2cm,lmargin=2cm,rmargin=2cm}

%\usepackage{graphicx}

\usepackage{fontspec}
\setmonofont{FreeMono}

\setlength{\parindent}{0cm}
\setlength{\parskip}{0.5em}

\usepackage{textcomp}

\usepackage{hyperref}
\usepackage{url}
\usepackage{xcolor}

\usepackage{minted}
\newminted{cpp}{breaklines,fontsize=\footnotesize}
\newminted{gnuplot}{breaklines,fontsize=\footnotesize}
\newminted{text}{breaklines,fontsize=\footnotesize}

\definecolor{mintedbg}{rgb}{0.95,0.95,0.95}
\usepackage{mdframed}

\BeforeBeginEnvironment{minted}{\begin{mdframed}[backgroundcolor=mintedbg]}
\AfterEndEnvironment{minted}{\end{mdframed}}

\title{
MI2101\\
Praktikum Teknik Komputasi\\
Modul 3}
\author{Fadjar Fathurrahman}
\date{2018}

\begin{document}
\maketitle

\section{Tujuan}
\begin{itemize}
\item Mampu membuat program C++ sederhana dengan memanfaatkan subprogram
\end{itemize}

\section{Perangkat lunak yang diperlukan}
\begin{itemize}
\item Linux OS
\item CodeBlocks yang telah dikonfigurasi untuk kompiler GNU C/C++
\item Terminal emulator dengan \texttt{bash} sebagai shell (baris perintah)
\item Editor teks seperti \texttt{gedit}
\item Perangkat lunak untuk membuat plot/grafik:
\begin{itemize}
\item Spreadsheet program seperti \textsf{Microsoft Excel} atau \textsf{Libre Office Calc}
\item \textsf{Gnuplot}
\end{itemize}
\end{itemize}

\section{Akar persamaan kuadrat dengan menggunakan subprogram (fungsi)}
Pada bagian ini kita akan kembali meninjau program untuk menghitung akar
persaman kuadrat. Program ini akan kita ubah sedikit, yaitu dengan memanfaatkan
subprogram (fungsi). Kita akan membagi langkah-langkah yang ada pada
program terdahulu dengan menggunakan fungsi.
Program lebih panjang daripada program sebelumnya (tanpa menggunakan fungsi)
karena bertujuan untuk menguji pemahaman Anda tentang fungsi dalam bahasa C++.

\begin{cppcode}
#include <iostream>
#include <cmath>

using namespace std;

void selamat_datang();
float baca_input_a();
float baca_input_b();
float baca_input_c();
void tampilkan_abc(float a, float b, float c);
float hitung_diskriminan(float a, float b, float c);
void hitung_dan_tampilkan_akar(float a, float b, float c);

int main()
{
  selamat_datang();

  float a, b, c;
  a = baca_input_a();
  b = baca_input_b();
  c = baca_input_c();

  tampilkan_abc(a, b, c);

  hitung_dan_tampilkan_akar(a, b, c);
  
  return 0;
}

void selamat_datang()
{
  cout << "Selamat datang" << endl;
  cout << endl;
  cout << "Saya adalah program untuk menghitung akar-akar persamaan kuadrat" << endl;
  cout << endl;
  cout << "a*x^2 + b*x + c = 0" << endl;
  cout << endl;
  cout << "Copyright: Jaka Sembung (2018)" << endl;
  cout << endl;
}

float baca_input_a()
{
  float a;
  for(;;) {
    cout << "Masukkan nilai a: ";
    cin >> a;
    if( fabs(a) <= 0 ) {
      cout << "Nilai yang Anda masukkan tidak valid" << endl;
      cout << "Mohon masukkan nilai a yang valid (a /= 0)" << endl;
    }
    else break;
  }
  return a;
}

// mirip dengan baca_input_a(), namun tanpa pengecekan nilai
float baca_input_b()
{...}

float baca_input_c()
{...}

void tampilkan_abc(float a, float b, float c)
{
  cout << endl;
  cout << "Berikut ini adalah nilai a, b, dan c yang telah saya baca" << endl;
  cout << endl;
  cout << "a = " << a << endl;
  cout << "b = " << b << endl;
  cout << "c = " << c << endl;
  cout << endl;
}

void hitung_dan_tampilkan_akar(float a, float b, float c)
{
  ...
  D = hitung_diskriminan(a,b,c);
  ...
  if( D >= 0.0 ) {
    ...
  }
  else {
    ...
  }
}

float hitung_diskriminan(float a, float b, float c)
{ ... }
\end{cppcode}


\subsection{Pertanyaan dan tugas}

\begin{enumerate}
\item Jelaskan fungsi/subprogram yang digunakan pada program.
\item Sebutkan apa yang menjadi input dan output dari setiap fungsi tersebut.
\item Misalkan deklarasi fungsi \texttt{hitung\_diskriminan} diubah menjadi
\begin{cppcode}
float hitung_diskriminan(float coef_a, float coef_b, float coef_c);
\end{cppcode}
Bagian mana saja yang harus diubah ?
%
\item Perhatikan program berikut ini.
\begin{cppcode}
#include <iostream>

using namespace std;

void func1(int a, int b);
void func2(int& a, int& b);

int main()
{
  int a = 1, b = 3;

  cout << "a = " << a << endl;
  cout << "b = " << b << endl;

  func1(a, b);
  cout << "a = " << a << endl;
  cout << "b = " << b << endl;
 
  func2(a, b);
  cout << "a = " << a << endl;
  cout << "b = " << b << endl;
  return 0;
}

void func1(int a, int b)
{
  a = 33;
  b = 44;
}

void func2(int &a, int &b)
{
  a = 99;
  b = 88;
}
\end{cppcode}
%
Tuliskan keluaran dari program di atas. Jelaskan apa perbedaan
antara dua fungsi \texttt{func1} dan \texttt{func2}.
Apa yang terjadi jika pada fungsi \texttt{main} urutan pemanggilan fungsi
dibalik ? (\texttt{func2} kemudian baru \texttt{func1}).
Catatan: fungsi \texttt{func1} disebut menggunakan \textit{pass-by-value}
dan fungsi \texttt{func2} disebut menggunakan \textit{pass-by-reference}.
%
\item Tinjau kembali fungsi untuk mencari akar-akar persamaan kuadrat yang
telah dibuat pada bagian awal modul ini. Kita akan mempersingkat fungsi
input yang awalnya berjumlah tiga fungsi, yaitu \texttt{baca\_input\_a},
\texttt{baca\_input\_b}, dan \texttt{baca\_input\_c} menjadi satu fungsi
yang dideklarasikan sebagai berikut.
\begin{cppcode}
void baca_input(float &a, float &b, float &c);
\end{cppcode}
Dengan menggunakan \textit{pass-by-reference}
buatlah implementasi dari fungsi tersebut dan gunakan dalam
program perhitungan akar-akar persamaan kuadrat.
\end{enumerate}





\end{document}
