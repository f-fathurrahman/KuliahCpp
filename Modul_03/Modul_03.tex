\documentclass[a4paper,11pt]{extarticle}
\usepackage[a4paper]{geometry}
\geometry{verbose,tmargin=2cm,bmargin=2cm,lmargin=2cm,rmargin=2cm}

%\usepackage{graphicx}

\usepackage{fontspec}
\setmonofont{FreeMono}

\setlength{\parindent}{0cm}
\setlength{\parskip}{0.5em}

\usepackage{textcomp}

\usepackage{hyperref}
\usepackage{url}
\usepackage{xcolor}

\usepackage{minted}
\newminted{cpp}{breaklines,fontsize=\small}
\newminted{gnuplot}{breaklines,fontsize=\small}
\newminted{text}{breaklines,fontsize=\small}

\definecolor{mintedbg}{rgb}{0.95,0.95,0.95}
\usepackage{mdframed}

\BeforeBeginEnvironment{minted}{\begin{mdframed}[backgroundcolor=mintedbg]}
\AfterEndEnvironment{minted}{\end{mdframed}}

\title{
MI2102\\
Praktikum Teknik Komputasi\\
Modul 3}
\author{Fadjar Fathurrahman}
\date{2018}

\begin{document}
\maketitle

\section{Tujuan}
\begin{itemize}
\item Mampu membuat program C++ sederhana dengan memanfaatkan subprogram
\end{itemize}

\section{Perangkat lunak yang diperlukan}
\begin{itemize}
\item Linux OS
\item CodeBlocks yang telah dikonfigurasi untuk kompiler GNU C/C++
\item Terminal emulator dengan \texttt{bash} sebagai shell (baris perintah)
\item Editor teks seperti \texttt{gedit}
\item Perangkat lunak untuk membuat plot/grafik:
\begin{itemize}
\item Spreadsheet program seperti \textsf{Microsoft Excel} atau \textsf{Libre Office Calc}
\item \textsf{Gnuplot}
\end{itemize}
\end{itemize}

\section{Akar persamaan kuadrat dengan menggunakan subprogram (fungsi)}
Pada bagian ini kita akan kembali meninjau program untuk menghitung akar
persaman kuadrat. Program ini akan kita ubah sedikit, yaitu dengan memanfaatkan
subprogram (fungsi). Kita akan membagi langkah-langkah yang ada pada
program terdahulu dengan menggunakan fungsi.
Program lebih panjang daripada program sebelumnya (tanpa menggunakan fungsi)
karena bertujuan untuk menguji pemahaman Anda tentang fungsi dalam bahasa C++.



\section{Latihan}



\end{document}
