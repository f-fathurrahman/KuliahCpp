\documentclass[a4paper,11pt]{extarticle}
\usepackage[a4paper]{geometry}
\geometry{verbose,tmargin=2cm,bmargin=2cm,lmargin=2cm,rmargin=2cm}

\usepackage{fontspec}
\setmonofont{FreeMono}

\setlength{\parindent}{0cm}
\setlength{\parskip}{0.5em}

\usepackage{textcomp}

\usepackage{amsmath}
\usepackage{hyperref}
\usepackage{url}
\usepackage{xcolor}

\usepackage{minted}
\newminted{cpp}{breaklines,fontsize=\footnotesize}
\newminted{gnuplot}{breaklines,fontsize=\footnotesize}
\newminted{text}{breaklines,fontsize=\footnotesize}

\definecolor{mintedbg}{rgb}{0.95,0.95,0.95}
\usepackage{mdframed}

\BeforeBeginEnvironment{minted}{\begin{mdframed}[backgroundcolor=mintedbg]}
\AfterEndEnvironment{minted}{\end{mdframed}}

\title{
MI2101\\
Praktikum Teknik Komputasi\\
Modul 4}
\author{Fadjar Fathurrahman}
\date{2018}

\begin{document}
\maketitle

\section{Tujuan}
\begin{itemize}
\item Mampu membuat program C++ sederhana dengan memanfaatkan array
\end{itemize}

\section{Perangkat lunak yang diperlukan}
\begin{itemize}
\item Linux OS
\item CodeBlocks yang telah dikonfigurasi untuk kompiler GNU C/C++
\item Terminal emulator dengan \texttt{bash} sebagai shell (baris perintah)
\item Editor teks seperti \texttt{gedit}
\end{itemize}

\section{Menghitung rata-rata}

Buatlah program untuk menghitung rata-rata dari nilai yang ada pada
suatu array. Gunakan fungsi dengan prototip sebagai berikut.
\begin{cppcode}
float hitung_rata2( int Ndata, float *data )
\end{cppcode}

Untuk menguji fungsi yang Anda buat, lengkapi program berikut.
\begin{cppcode}
#include <iostream>

using namespace std;

float hitung_rata2( int Ndata, float *data );

int main()
{
  const int N = 10;

  float nilai[N] = {40.9, 45.3, 55.3, 55.1, 33.1,
                    11.7, 34.1, 13.2, 33.3, 34.1};
  
  cout << "Nilai rata = " << hitung_rata2(N, nilai) << endl;

  return 0;
}

float hitung_rata2( int Ndata, float *data )
{
  // tuliskan fungsi Anda di sini
}
\end{cppcode}


\end{document}
