\documentclass[a4paper,11pt]{extarticle}
\usepackage[a4paper]{geometry}
\geometry{verbose,tmargin=2cm,bmargin=2cm,lmargin=2cm,rmargin=2cm}

\usepackage{fontspec}
\setmonofont{FreeMono}

\setlength{\parindent}{0cm}
\setlength{\parskip}{0.5em}

\usepackage{textcomp}

\usepackage{amsmath}
\usepackage{hyperref}
\usepackage{url}
\usepackage{xcolor}

\usepackage{minted}
\newminted{cpp}{breaklines,fontsize=\footnotesize}
\newminted{gnuplot}{breaklines,fontsize=\footnotesize}
\newminted{text}{breaklines,fontsize=\footnotesize}

\definecolor{mintedbg}{rgb}{0.95,0.95,0.95}
\usepackage{mdframed}

\BeforeBeginEnvironment{minted}{\begin{mdframed}[backgroundcolor=mintedbg]}
\AfterEndEnvironment{minted}{\end{mdframed}}

\title{
MI2101\\
Praktikum Teknik Komputasi\\
Modul 4}
\author{Fadjar Fathurrahman}
\date{2018}

\begin{document}
\maketitle

\section{Tujuan}
\begin{itemize}
\item Mampu membuat program C++ sederhana dengan memanfaatkan array
\end{itemize}

\section{Perangkat lunak yang diperlukan}
\begin{itemize}
\item Linux OS
\item CodeBlocks yang telah dikonfigurasi untuk kompiler GNU C/C++
\item Terminal emulator dengan \texttt{bash} sebagai shell (baris perintah)
\item Editor teks seperti \texttt{gedit}
\end{itemize}

\section{Menghitung rata-rata}

Buatlah program untuk menghitung rata-rata dari nilai yang ada pada
suatu array. Gunakan fungsi dengan prototip sebagai berikut.
\begin{cppcode}
float hitung_rata2( int Ndata, float *data )
\end{cppcode}

Untuk menguji fungsi yang Anda buat, lengkapi program berikut.
\begin{cppcode}
#include <iostream>

using namespace std;

float hitung_rata2( int Ndata, float *data );

int main()
{
  const int N = 10;

  float nilai[N] = {40.9, 45.3, 55.3, 55.1, 33.1,
                    11.7, 34.1, 13.2, 33.3, 34.1};
  
  cout << "Nilai rata = " << hitung_rata2(N, nilai) << endl;

  return 0;
}

float hitung_rata2( int Ndata, float *data )
{
  // tuliskan fungsi Anda di sini
}
\end{cppcode}


\section{Mengolah nilai akhir suatu mata kuliah}
Anda diberikan tugas untuk membuat program untuk mengolah nilai akhir mahasiswa
dari suatu mata kuliah. Ada tiga komponen nilai yang akan dihitung, yaitu nilai UTS,
nilai UAS, dan nilai praktikum. Nilai akhir dihitung dengan persamaan sebagai
berikut:
\begin{equation}
\mathrm{NilaiAkhir} = 0.3\mathrm{NilaiUTS} + 0.3\textrm{NilaiUAS} + 0.4\textrm{NilaiPraktikum}
\end{equation}
%
Selain itu, akan ditentukan juga nilai akhir dalam bentuk huruf, dengan aturan sebagai
berikut.

{\centering
\begin{tabular}{|c|c|}
\hline
Nilai akhir & Nilai huruf \\
\hline
$90 < \mathrm{NilaiAkhir} \leq 100$ & A \\
$80 < \mathrm{NilaiAkhir} \leq 90$ & AB \\
$70 < \mathrm{NilaiAkhir} \leq 80$ & B \\
$60 < \mathrm{NilaiAkhir} \leq 70$ & BC \\
$50 < \mathrm{NilaiAkhir} \leq 60$ & C \\
$\mathrm{NilaiAkhir} \leq 50$ & E \\
\hline
\end{tabular}
\par}



Kode program:
\begin{cppcode}
#include <iostream>
#include <string>

using namespace std;

void input_data(
  int NDATA,
  string *nama, string *NIM, float *uts, float *uas, float *praktikum
);

void hitung_nilai_akhir(
  int Ndata,
  float *uts, float *uas, float *praktikum,
  float *nilai_akhir
);

void tampilkan_data(
  int idata,
  string *nama, string *NIM, float *uts, float *uas, float *praktikum,
  float *nilai_akhir
);

int main()
{
  const int NDATA = 3;
  
  string nama[NDATA];
  string NIM[NDATA];
  float uts[NDATA], uas[NDATA], praktikum[NDATA];
  float nilai_akhir[NDATA];

  input_data( NDATA, nama, NIM, uts, uas, praktikum );

  hitung_nilai_akhir( NDATA, uts, uas, praktikum, nilai_akhir );

  for(int i = 0; i < NDATA; i++) {
    tampilkan_data( i, nama, NIM, uts, uas, praktikum, nilai_akhir );
  }

  return 0;

}

void input_data(
  int NDATA,
  string *nama, string *NIM, float *uts, float *uas, float *praktikum
)
{
  for(int i = 0; i < NDATA; i++) {
    cout << endl;
    cout << "---------------" << endl;
    cout << "Entri data:" << i+1 << endl;
    cout << "---------------" << endl;
    cout << "Masukkan data" << endl;
    
    // handle inputs here ... (nama, nim, uts, uas, praktikum)
    // ............
  }
}

void tampilkan_data(
  int i,
  string *nama, string *NIM, float *uts, float *uas, float *praktikum,
  float *nilai_akhir
)
{
  cout << endl;
  cout << "--------------------" << endl;
  cout << "Tampilan data: " << i+1 << endl;
  cout << "--------------------" << endl;
  cout << "Nama            : " << nama[i] << endl;
  cout << "NIM             : " << NIM[i] << endl;
  cout << "Nilai UTS       : " << uts[i] << endl;
  cout << "Nilai UAS       : " << uas[i] << endl;
  cout << "Nilai praktikum : " << praktikum[i] << endl;
  cout << "------------------" << endl;
  cout << "Nilai akhir     : " << nilai_akhir[i];
  cout << " (" << hitung_nilai_huruf(nilai_akhir[i]) << ")" << endl;
}

void hitung_nilai_akhir(
  int Ndata,
  float *uts, float *uas, float *praktikum,
  float *nilai_akhir
)
{
  // lengkapi .....
}


string hitung_nilai_huruf( float nilai_akhir )
{
  // lengkapi .....
}
\end{cppcode}


\end{document}
