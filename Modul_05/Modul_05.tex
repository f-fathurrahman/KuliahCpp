\documentclass[a4paper,11pt]{extarticle}
\usepackage[a4paper]{geometry}
\geometry{verbose,tmargin=2cm,bmargin=2cm,lmargin=2cm,rmargin=2cm}

\usepackage{fontspec}
\setmonofont{FreeMono}

\setlength{\parindent}{0cm}
\setlength{\parskip}{0.5em}

\usepackage{textcomp}

\usepackage{amsmath}
\usepackage{hyperref}
\usepackage{url}
\usepackage{xcolor}

\usepackage{minted}
\newminted{cpp}{breaklines,fontsize=\footnotesize}
\newminted{gnuplot}{breaklines,fontsize=\footnotesize}
\newminted{text}{breaklines,fontsize=\footnotesize}

\definecolor{mintedbg}{rgb}{0.95,0.95,0.95}
\usepackage{mdframed}

\BeforeBeginEnvironment{minted}{\begin{mdframed}[backgroundcolor=mintedbg]}
\AfterEndEnvironment{minted}{\end{mdframed}}

\title{
MI2101\\
Praktikum Teknik Komputasi\\
Modul 5}
\author{Fadjar Fathurrahman}
\date{2018}

\begin{document}
\maketitle

\section{Tujuan}
\begin{itemize}
\item Mampu membuat program C++ sederhana dengan memanfaatkan \texttt{struct}
\end{itemize}

\section{Perangkat lunak yang diperlukan}
\begin{itemize}
\item Linux OS
\item CodeBlocks yang telah dikonfigurasi untuk kompiler GNU C/C++
\item Terminal emulator dengan \texttt{bash} sebagai shell (baris perintah)
\item Editor teks seperti \texttt{gedit}
\end{itemize}

\section{Pengenalan \texttt{struct}}
\texttt{struct} dapat dianggap sebagai tipe data bentukan atau komposit, yang terdiri
dari beberapa data lain (baik tipe data primitif atau tipe data bentukan lain).
Sintaks untuk mendeklarasi sebuah \texttt{struct} dalam C++ adalah
sebagai berikut:
\begin{cppcode}
struct NamaStruct {
  tipeData1 data1;
  tipeData2 data2;
  // ... dan seterusnya.
};
\end{cppcode}

Kita akan menggunakan masalah dari Modul 4 untuk mempelajari mengenai
\texttt{struct}.

Pada program berikut ini kita akan mendefisikan sebuah \texttt{struct} dengan
nama \texttt{Mahasiswa} yang terdiri dari:
\begin{itemize}
\item nama mahasiswa
\item NIM
\item Nilai UTS
\item Nilai UAS
\item Nilai praktikum
\item Nilai akhir
\end{itemize}
Nilai akhir bergantung dari nilai UTS, UAS, dan nilai praktikum sebagaimana
yang telah diberikan pada Modul 4.

Pada program berikut ini, untuk sementara kita tidak menggunakan nilai akhir.

\begin{cppcode}
#include <iostream>
#include <string>

using namespace std;

struct Mahasiswa
{
  string nama;
  string NIM;
  float uts;
  float uas;
  float praktikum;
};

void info_mahasiswa(Mahasiswa m);

int main()
{
  Mahasiswa a;
  a.nama = "Jojo";
  a.NIM = "A018003";
  a.uts = 80;
  a.uas = 99;
  a.praktikum = 78;

  info_mahasiswa(a);

  // menggunakan initializer list
  Mahasiswa b;
  b = {"Johann", "A018004", 81, 70, 65};
  info_mahasiswa(b);

  return 0;
}

void info_mahasiswa(Mahasiswa m)
{
  cout << endl;
  cout << "Nama            : " << m.nama << endl;
  cout << "NIM             : " << m.NIM << endl;
  cout << "Nilai UTS       : " << m.uts << endl;
  cout << "Nilai UAS       : " << m.uas << endl;
  cout << "Nilai praktikum : " << m.praktikum << endl;
}
\end{cppcode}

\subsection*{Pertanyaan dan tugas}

\begin{enumerate}
%
\item Tuliskan keluaran dari program.
%
\item Jelaskan maksud sintaks berikut.
\begin{cppcode}
Mahasiswa a;
\end{cppcode}
%
\item Jelaskan maksud sintaks berikut.
\begin{cppcode}
Mahasiswa Jim;
Jim.nama = "Jim Carey";
Jim.NIM = "A018099";
\end{cppcode}
%
\item Misalkan Anda hanya memberikan nilai pada \texttt{nama} dan \texttt{NIM}
kemudian memanggil
subrutin \texttt{info\_mahasiswa}, apakah yang akan ditampilkan ?
Bagaimana jika nama dan NIM tidak diberikan?
%
\item Misalkan fungsi \texttt{main()} pada program di atas diganti
sebagai berikut.
\begin{cppcode}
int main()
{
  Mahasiswa Tom;
  Tom.nama = "Tom Cruise";
  Tom.nim = "A018100";
  Tom.uts = 77;
  Tom.uas = 66;
  Tom.praktikum = 55;
  info_mahasiswa(Mahasiswa a);
}
\end{cppcode}
Apa yang terjadi ? Perbaiki program jika ada kesalahan pada program.
%
\item Kembangkan program di atas dengan menambahkan \texttt{nilai\_akhir} pada
\texttt{struct Mahasiswa} sehingga:
\begin{cppcode}
struct Mahasiswa
{
  string nama;
  string NIM;
  float uts;
  float uas;
  float praktikum;
  float nilai_akhir.
};
\end{cppcode}
%
Kemudian implementasikan fungsi hitung nilai akhir dengan prototip (deklarasi)
sebagai berikut. (Gunakan persamaan pada Modul 4 untuk menghitung nilai akhir)
\begin{cppcode}
void hitung_nilai_akhir(Mahasiswa &m);
\end{cppcode}
Mengapa harus menggunakan \texttt{(Mahasiswa \&m)}?
%
Lengkapi juga subrutin \texttt{info\_mahasiswa} sebagai berikut.
\begin{cppcode}
void info_mahasiswa(Mahasiswa m)
{
  cout << endl;
  cout << "Nama            : " << m.nama << endl;
  cout << "NIM             : " << m.NIM << endl;
  cout << "Nilai UTS       : " << m.uts << endl;
  cout << "Nilai UAS       : " << m.uas << endl;
  cout << "Nilai praktikum : " << m.praktikum << endl;
  cout << "---------------------------" << endl;
  cout << "Nilai akhir     : " << ....; // isi titik-titk
  cout << " (" << hitung_nilai_huruf(....) << ")" << endl; // isi titik-titik
}
\end{cppcode}
Gunakan fungsi \texttt{hitung\_nilai\_huruf} seperti pada Modul 4.
%
Contoh keluaran program adalah sebagai berikut.
\begin{textcode}
Nama            : Jojo
NIM             : A018003
Nilai UTS       : 80
Nilai UAS       : 99
Nilai praktikum : 78
---------------------------
Nilai akhir     : 84.9 (AB)

Nama            : Johann
NIM             : A018004
Nilai UTS       : 81
Nilai UAS       : 70
Nilai praktikum : 65
---------------------------
Nilai akhir     : 71.3 (B)
\end{textcode}
\end{enumerate}



\section{Array dari \texttt{struct}}
Sebagaimana tipe data primitif, array juga dapat dibangun dari \texttt{struct}.

Contoh program:
\begin{cppcode}
#include <iostream>
#include <string>

using namespace std;

struct Mahasiswa
{
  string nama;
  string NIM;
  float uts;
  float uas;
  float praktikum;
};

void info_mahasiswa(Mahasiswa a);

int main()
{
  int NDATA = 3;
  Mahasiswa *daftar_mhs;

  daftar_mhs = new Mahasiswa[NDATA];

  daftar_mhs[0].nama = "Jojo";
  daftar_mhs[0].NIM = "A018003";
  daftar_mhs[0].uts = 80;
  daftar_mhs[0].uas = 99;
  daftar_mhs[0].praktikum = 78;
  
  // isi data lain jika diperlukan

  info_mahasiswa(daftar_mhs[0]);
  info_mahasiswa(daftar_mhs[1]);
  info_mahasiswa(daftar_mhs[2]);

  delete[] daftar_mhs;

  return 0;
}

void info_mahasiswa(Mahasiswa m)
{
  cout << endl;
  cout << "Nama            : " << m.nama << endl;
  cout << "NIM             : " << m.NIM << endl;
  cout << "Nilai UTS       : " << m.uts << endl;
  cout << "Nilai UAS       : " << m.uas << endl;
  cout << "Nilai praktikum : " << m.praktikum << endl;
}
\end{cppcode}

\subsection*{Tugas}
Kembangkan program seperti tugas pada Modul 4 dengan menggunakan \texttt{struct}.
Anda dapat menambahkan fungsi-fungsi yang lain jika diperlukan.


\end{document}
