\documentclass[a4paper,11pt]{extarticle}
\usepackage[a4paper]{geometry}
\geometry{verbose,tmargin=2cm,bmargin=2cm,lmargin=2cm,rmargin=2cm}

\usepackage{fontspec}
\setmonofont{FreeMono}

\setlength{\parindent}{0cm}
\setlength{\parskip}{0.5em}

\usepackage{textcomp}

\usepackage{hyperref}
\usepackage{url}
\usepackage{xcolor}

\usepackage{minted}
\newminted{cpp}{breaklines,fontsize=\small}
\newminted{text}{breaklines,fontsize=\small}

\definecolor{mintedbg}{rgb}{0.95,0.95,0.95}
\usepackage{mdframed}

\BeforeBeginEnvironment{minted}{\begin{mdframed}[backgroundcolor=mintedbg]}
\AfterEndEnvironment{minted}{\end{mdframed}}

\title{
MI2102\\
Praktikum Teknik Komputasi\\
Modul 2}
\author{Fadjar Fathurrahman}
\date{2018}

\begin{document}
\maketitle

\section{Tujuan}
\begin{itemize}
\item Simple user input
\item Percabangan dan perulangan
\item Membuat data plot dengan spreadsheet atau gnuplot
\end{itemize}

\section{Perangkat lunak yang diperlukan}
\begin{itemize}
\item Linux OS
\item CodeBlocks yang telah dikonfigurasi untuk kompiler GNU C/C++
\item Terminal emulator dengan \texttt{bash} sebagai shell (baris perintah)
\item Editor teks seperti \texttt{gedit}
\item Perangkat lunak untuk membuat plot/grafik:
\begin{itemize}
\item Spreadsheet program seperti \textsf{Microsoft Excel} atau \textsf{Libre Office Calc}
\item \textsf{Gnuplot}
\end{itemize}
\end{itemize}

\section{Akar persamaan kuadrat}
Pada modul sebelumnya kita telah membuat program sederhana untuk menghitung
akar-akar dari persamaan kuadrat
\begin{equation}
ax^2 + bx + c = 0
\label{eq:pers_kuadrat}
\end{equation}
$a, b, c$ adalah bilangan real, dengan batasan bahwa $D = b^2 - 4ac >= 0$.
Pada bagian ini, kita akan memperbaiki program sebelumnya dengan
membolehkan kasus $D < 0$. Pada kasus ini, akar-akar dari persamaan
\ref{eq:pers_kuadrat} dapat dinyatakan sebagai:
\begin{equation}
x_{1,2} = -\frac{b}{2a} \pm \frac{\sqrt{-D}}{2a}\imath
\end{equation}
Dari persamaan di atas dapat dilihat bahwa $x_{1}$ dan $x_{2}$ adalah
pasangan konjugat kompleks.

\subsection{Tugas: akar-akar real dan imajiner}

Buatlah program untuk menghitung akar-akar dari persamaan kuadrat
seperti pada Modul 1, namun juga memperhitungkan kasus diskriminan negatif
(akar-akar kompleks).
Anda dapat menggunakan konstruksi \texttt{if-else} pada C++.
Program berikut ini dapat Anda lengkapi sebagai panduan.

\begin{cppcode}
#include <iostream>
#include <cmath>

using namespace std;

int main()
{
  float a, b, c;
  // berikan nilai a, b, dan c di sini
  a = ...;
  b = ...;
  c = ...;
  // Tampilkan pesan ke layar
  cout << "Mencari akar-akar persamaan kuadrat" << endl;
  cout << endl;
  cout << "a*x^2 + b*x + c = 0" << endl;
  cout << endl;
  cout << "a = " << a << endl;
  cout << "b = " << b << endl;
  cout << "c = " << c << endl;
  cout << endl;
  // Hitung diskriminan di sini
  float D;
  D = ....;
  // Tampilkan nilai diskriminan
  ....
  // Deklarasi variabel
  ....
  if( D >= 0.0 ) { // akar real
    x1 = ....;
    x2 = ....;
    cout << "Akar-akar real:" << endl;
    .... // tampilkan x1 dan x2
  }
  else { // akar imajiner
    ....
    cout << "Akar-akar imajiner:" << endl;
    ....
  }
  return 0;
}
\end{cppcode}

Contoh keluaran program di atas untuk kasus akar-akar real.
\begin{textcode}
Mencari akar-akar persamaan kuadrat

a*x^2 + b*x + c = 0

a = 2
b = 1
c = -4

D = 33

Akar-akar real:

x1 = 1.18614
x2 = -1.68614
\end{textcode}

Contoh keluaran program di atas untuk kasus akar-akar imajiner.
\begin{textcode}
Mencari akar-akar persamaan kuadrat

a*x^2 + b*x + c = 0

a = 2
b = 1
c = 4

D = -31

Akar-akar imajiner:

x1 = -0.25 + 1.39194i
x2 = -0.25 - 1.39194i
\end{textcode}


\subsection{Tugas: akar persamaan kuadrat dengan input dari pengguna}

Modifikasi program pada tugas sebelumnya agar dengan membaca input
nilai $a, b, c$ dari pengguna secara interaktif. Anda
dapat menggunakan pernyataan \texttt{cin} pada C++.

Contoh keluaran dari program:
\begin{textcode}
Masukkan nilai a: 1.0
Masukkan nilai b: 2.1
Masukkan nilai c: 8.0

Mencari akar-akar persamaan kuadrat

a*x^2 + b*x + c = 0

a = 1
b = 2.1
c = 8

D = -27.59

Akar-akar imajiner:

x1 = -1.05 + 2.62631i
x2 = -1.05 - 2.62631i
\end{textcode}


\section{Perulangan}



\end{document}
